\documentclass[a4paper, twocolumn]{article}
\usepackage[pdftex, hidelinks]{hyperref}

\usepackage{bm}
\usepackage[T1]{fontenc}
\usepackage[utf8]{inputenc}
\usepackage{algorithmic}
\usepackage{algorithm}
\usepackage{amsfonts}
\usepackage{amssymb}
\usepackage{courier}
\usepackage{booktabs}
\usepackage{graphicx}
\usepackage{listings}
\usepackage{mathtools}
\usepackage{amssymb}
\lstset{basicstyle=\footnotesize\ttfamily,
        breakatwhitespace = false,
        breaklines = true,
        keepspaces = true,
        language = R,
        showspaces = false,
        showstringspaces = false,
        belowcaptionskip = \bigskipamount,
        framerule = 0.80pt,
        frame = tb,
        belowskip = \bigskipamount,
        escapeinside={<@}{@>}}

\title{TDDE01 -- Machine Learning \\
       Group 9 Laboration Report 5}
\author{{Martin Estgren \texttt{<mares480>}} \\
        {Erik S. V. Jansson \texttt{<erija578>}} \\
        {Sebastian Maghsoudi \texttt{<sebma654>}} \\~\\
        {Linköping University (LiU), Sweden}}

\begin{document}
    \pagenumbering{arabic}
    \maketitle % Generate.

    Increasing the accuracy of \emph{weather forecasts} is an important task. We propose an estimator which produces the \emph{air temperature forecast} in \emph{Sweden}, given a \emph{latitude/longitude coordinate} and also \emph{date}. Some observations by \emph{SMHI}, taken from weather stations, have been given for training our estimator.

    By using a \emph{Nadaraya–Watson regression kernel}, we can estimate the temperatures \(\bm{y'}\). This is done by taking the \emph{kernels} \(k_\sigma(\bm{x}^{(i)}, \bm{x'})\) for each \(i^{th}\) data from the training set and using it as a \emph{weight} when considering the response variable \(\bm{y}^{(i)}\). Essentially, the kernel \(k_\sigma(\bm{x}^{(i)}, \bm{x'})\) will reduce \(\bm{y}^{(i)}\)'s significance in the \emph{total contribution} by giving less weight when the \(\bm{x^{(i)}}\) and \(\bm{x'}\) are further away (in some measure).

    We have used a \emph{Gaussian Radial Basis Function} as our \emph{kernel}, which is defined in Equation~\ref{eq:grbf} below. Note the parameter \(\sigma\), which can be considered as the \emph{spread} or \emph{width} of the kernel, and also \(\bm{x^{(i)}} - \bm{x'}\) which is the \emph{distance function}; giving our kernel the property of a \emph{similarity function} (because of \(e^{(\cdots)}\)).

    By using \(k_\sigma(\bm{x}^{(i)}, \bm{x'})\) in \emph{Nadaraya–Watson's} \(\bm{y'}\) estimator, shown in Equation~\ref{eq:nadaraya_watson}, we are essentially \emph{weighing} how important the contributions from \(\bm{y}^{(i)}\) are to \(\bm{y'}\), because \emph{similar} \(\bm{x}^{(i)}\) will give higher \(k_\sigma\). Further descriptions: see \emph{Friedman et al.\ }~\cite{friedman2009elements}.

    \begin{equation} \label{eq:grbf}
        k_\sigma(\bm{x}, \bm{x'}) = \mathrm{exp}\bigg(\frac{- \, {\left\Vert(\bm{x} - \bm{x'}) \right\Vert}^2}
                                                           {2\sigma^2 \; \{\sigma \approx h\}}\bigg)
    \end{equation}

    \begin{equation} \label{eq:nadaraya_watson}
        \bm{y'}(\bm{x}, \bm{x'}) = \frac{\sum_n{\bm{y}^{(i)}k_\sigma(\bm{x}^{(i)}, \bm{x'})}}
                                               {\sum_n{k_\sigma(\bm{x}^{(i)}, \bm{x'})}}
    \end{equation}

    We now describe the functions which were used to determine the distance/similarity for the kernels.

    Below follows the applied \emph{distance functions}, which give the measured distance between a pair of \emph{locations}, \emph{times of the day}, and also \emph{dates of year}. These distance and kernel functions can be found in Listing~\ref{lst:forecast}, where they follow similarly as described.

    \begin{equation*} \label{eq:location}
        d_l = r\, \mathrm{hav}^{-1}(h),\; \mathrm{hav}(\varphi) = \frac{1 - \cos\varphi}{2}
    \end{equation*}

    \begin{equation*} \label{eq:time}
        d_t = \begin{cases}
            |x - y| & |x - y| < (x + y) \bmod 24\\
            (x + y) \bmod 24 & |x - y| \geq (x + y) \bmod 24
        \end{cases}
    \end{equation*}

    \begin{equation*} \label{eq:day}
        d_d = \begin{cases}
            |x - y| & |x - y| < (x + y) \bmod 365\\
            (x + y) \bmod 365 & |x - y| \geq (x + y) \bmod 365
        \end{cases}
    \end{equation*}

    Finally, we have the \emph{three different kernels} which contribute in unison to the final \emph{forecast kernel}. Each of these use Equation~\ref{eq:grbf} with their respective \emph{distance functions} and also \emph{kernel width} \(h\) (\(\sigma\) here).

    \section*{Contributions}

        \begin{itemize}
            \item{\textbf{Martin Estgren:} ...}
            \item{\textbf{Erik S. V. Jansson:} ...}
            \item{\textbf{Sebastian Maghsoudi:} ...}
        \end{itemize}

    \nocite{*} % No warnings.
    \bibliographystyle{alpha}
    \bibliography{report}
    \onecolumn \appendix
    \section*{Appendix}

        ...

\end{document}
